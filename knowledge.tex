\documentclass[a4paper,12pt]{ctexart}
\usepackage{amsmath}
\usepackage{graphicx}
\usepackage{float}
\usepackage{booktabs}

\title{数据驱动下磁性元件的磁芯损耗建模}
\author{}
\date{}

\begin{document}

\maketitle

\section{背景}

磁芯损耗是磁性材料在高频交变磁通作用下产生的功率损耗(本题中损耗都是指功率损耗),由于高频磁性材料(如铁氧体、合金磁粉芯、非晶/纳米晶等)本身的微观结构复杂,且其损耗与工作频率、磁通密度、励磁波形、工作温度、磁芯材料等诸多因素有关,并呈现复杂的非线性和相互关联性。现有的磁芯材料损耗模型与实际应用的需求有较大差异。

目前,磁芯损耗模型主要分为两大类:损耗分离模型和经验计算模型。

\subsection{磁芯损耗分离模型}
磁芯损耗可分成 3 个部分:磁滞损耗、涡流损耗和剩余损耗。计算磁芯损耗的方法称为损耗分离模型,其计算公式如下:

\begin{equation}
P_{\text{core}} = P_h + P_{cl} + P_e
\end{equation}

其中: $P_{\text{core}}$ 为总的磁芯损耗密度, $P_h$ 为磁滞损耗, $P_{cl}$ 为涡流损耗, $P_e$ 为剩余损耗。

\subsection{磁芯损耗经验计算模型}

斯坦麦茨方程(Steinmetz-equation(SE))是最著名的经验计算模型之一。在正弦波励磁下,磁芯损耗计算公式如下:

\begin{equation}
P = k_1 f^{\alpha_1} B_m^{\beta_1}
\end{equation}

其中:$P$ 为磁芯损耗,$f$ 是频率,$B_m$ 是磁通密度的峰值;$k_1$、$\alpha_1$、$\beta_1$ 是根据实验数据拟合的系数,通常 $1<\alpha_1<3$,$2<\beta_1<3$。公式表明了单位体积的磁芯损耗 $P$ 取决于频率 $f$ 和磁通密度峰值 $B_m$ 的幂函数。SE 方程仅适用于正弦波励磁,并且不同的磁芯材料和工况(工况指的是磁性材料所处的不同工作环境,包括温度、频率、励磁波形等),其系数 $k_1$、$\alpha_1$、$\beta_1$ 未必相同(更详细内容请参阅参考文献后备注 1)。

由上述可以看出,磁芯损耗与温度、材料、频率、磁通密度的峰值有关。但影响磁芯损耗的因素较多,现阶段,鲜有普遍适用并且精度高的模型,这使得业内在使用磁性元件时无法对磁芯损耗做出精确的评估,进而影响到对功率变换器效率的评估。因此,希望基于数据驱动,建立一个高精度并且普遍适用于各种工况的磁芯损耗模型成为亟待解决的问题(参考文献后有备注 1:磁芯损耗模型的相关知识)。

\section{问题描述}

为解决磁性元件磁芯材料损耗精确计算问题,通过实测磁性元件在给定工况(不同温度、频率、磁通密度)下磁芯材料损耗的数据,通数学建模(或算法)方法,建立功率磁性元件的磁芯材料损耗模型,并且将其预测磁性元件在其他工况中的磁芯损耗,检测模型精确度。

\subsection{实验场景和采集数据}

磁芯损耗的测量目前一般采用交流功率法,如图 \ref{fig:core_loss_measurement} 所示。被测磁芯一般采用圆环形($l_e$ 是平均磁路长度,$A_e$ 是磁芯截面积),在磁芯上均匀绕制励磁绕组和感应绕组($N_1$ 和 $N_2$ 分别是励磁绕组和感应绕组的匝数,一般取 $N_1 = N_2$),信号发生器产生给定频率 $f$ (周期 $T=1/f$)的正弦波或者其他波形,经过高频功率放大器高频作为励磁源施加到励磁绕组上,根据安培环路定律,绕组励磁电流 $i(t)$ 在磁芯上产生磁场强度 $H(t)$(磁场强度是指单位电流元在磁场中所受到的洛伦兹力),磁场强度作用于导磁材料磁芯上产生磁通密度 $B(t)$,根据电磁感应定律,磁芯中交变的磁通则在感应绕组上产生感应电压 $u(t)$,进一步通过采集被测磁芯绕组的励磁电流 $i(t)$,从而得到磁场强度 $H(t)$ 和磁通密度 $B(t)$,利用式(3)计算出励磁源的输出功率(单位体积),即磁芯损耗密度 $P$:

\begin{equation}
P = \frac{1}{T} \int_0^T u(t) i(t) dt = \frac{1}{T} \int_0^T H(t) dB(t)
\end{equation}

\begin{figure}[H]
\centering
\includegraphics[width=0.5\textwidth]{core_loss_measurement}
\caption{双绕组法测量磁芯损耗}
\label{fig:core_loss_measurement}
\end{figure}

由式(3)也可知:一个励磁周期内的单位体积磁芯损耗就等于 $B-H$ 磁滞回线的面积,如图 \ref{fig:BH_loop} 所示。

\begin{figure}[H]
\centering
\includegraphics[width=0.5\textwidth]{BH_loop}
\caption{B-H 磁滞回线}
\label{fig:BH_loop}
\end{figure}

\subsection{数据说明}

\begin{enumerate}
    \item 解压附件(数据).rar,附件一为训练集,有4个数据表,分别表示来自4种不同磁芯材料所测的数据(由于磁芯材料的复杂性,我们仅用材料1、材料2、材料3、材料4来表示不同材料),4个数据表结构相同,其中:
    \begin{enumerate}
        \item 第1列是温度,取4个值:25、50、70、90,单位:摄氏度(°C);
        \item 第2列是频率,取值范围:50000—500000,单位:赫兹(Hz);
        \item 第3列是磁芯损耗,单位:每立方米瓦特(W/m³);
        \item 第4列是励磁波形类型:正弦波、三角波和梯形波;
        \item 第5—1029列是磁通密度,共1024个采样点(一个周期时间内,相同间距采样),单位:特斯拉(T)。
    \end{enumerate}
    \item 附件二、附件三为测试集,其中附件二:第1列是样本序号;第2列是温度;第3列是频率;第4列是磁芯材料;第5—1029列是磁通密度。
\end{enumerate}

\section{问题一:励磁波形分类}

励磁波形作为影响磁芯性能的核心要素之一,其形态深刻影响着磁芯的损耗特性。励磁波形的独特形状直接塑造了磁芯内部磁通的动态行为,不同的波形轮廓影响了磁通密度随时间的变化速率,导致其损耗特性呈现出显著差异。因此,准确识别出励磁波形,对于深入理解磁芯损耗机制、优化磁芯设计具有至关重要的价值。

励磁波形主要体现在磁通密度随时间变化的分布规律上,不同的励磁波形会导致磁通密度呈现出不同的增长、衰减或波动模式。请利用附件一中的磁通密度数据,首先分析磁通密度的分布特征及不同波形的形状特征,提取出反映磁通密度分布及波形的形状特征变量;然后利用这些特征变量建立分类模型,识别出励磁的三种波形,分析分类模型的合理性及有效性;并对附件二中的样本识别出相应波形,把分类结果填入附件四(Excel表格)中第2列,要求:

\begin{enumerate}
    \item 按样本序号填入相应的分类结果,只填数字,1 表示正弦波,2 表示三角波,3 表示梯形波,比如:附件二中第1个样品分类结果是三角波,在第2列样本序号为1对应行就填数字2;
    \item 结果填入附件四后,保留原文件名,以附件材料上传;
    \item 统计出附件二中三种波形的各自数量,呈现在论文正文中;
    \item 特别把附件二中样本序号为:1, 5, 15, 25, 35, 45, 55, 65, 75, 80 的分类结果,以表格形式呈现在论文正文中。
\end{enumerate}


\section{问题二:斯坦麦茨方程(Steinmetz-equation)修正}

在传统磁芯损耗模型中,斯坦麦茨方程(Steinmetz-equation)(公式(2))虽作为经典模型被广泛应用,却显著受限于其特定的适用条件,如:该方程主要针对正弦波形设计;对于不同种类的磁芯材料及工作温度的变化,SE 方程会造成较大的误差,这在实际工程应用中带来了诸多不便与复杂性。目前已经有针对非正弦波形下磁芯损耗模型进行修正(见公式(7)、(8))。

请通过分析斯坦麦茨方程(公式(2)),在同一种磁芯材料、正弦波形下,对于不同温度变化,磁芯损耗预测效果存在的差异性,构造一种可适用于不同温度变化的磁芯损耗修正方程(即在原斯坦麦茨方程基础上,增加温度这个因素,以适应不同温度变化,使磁芯损耗预测效果更好);并以附件一材料 1 中正弦波形的数据为例,分析你构造的修正方程与斯坦麦茨方程,他们预测磁芯损耗的效果(误差)哪个更好?

\section{问题三:磁芯损耗因素分析}

在磁性元件的设计与优化领域,磁芯损耗是一个核心指标,其大小直接关系到设备的效率与稳定性。在众多影响磁芯损耗的因素中,温度、励磁波形以及磁芯材料被公认为是最常见且比较重要的三大要素。为了精准提升磁性元件的性能,我们需要依托实验数据,深入剖析这三者如何独立或协同作用于磁芯损耗,并探索实现最低损耗的最优条件。

请根据附件一中的实验数据,通过数据分析技术,分析温度、励磁波形和磁芯材料这三个因素,是如何独立及协同影响着磁芯损耗(仅讨论两两之间协同影响);以及他们各自的影响程度;并给出这三个因素在什么条件下,磁芯损耗可能达到最小?

\section{问题四:基于数据驱动的磁芯损耗预测模型}

在磁芯损耗的研究领域中,尽管存在着众多传统模型(如文首“背景”所述),这些模型各自在不同的条件下展现了一定的应用价值,但普通面临精度不足或适用范围受限的挑战。当下,业界缺乏一个既广泛适用又能提供高精度预测结果的磁芯损耗模型,这直接制约了磁性元件设计中对损耗的精确评估,进而影响了整体功率变换器效率的有效预估。鉴于这一现状,业界对构建更为便捷、精准的数据驱动模型寄予了厚望,旨在开发出一个能够跨越不同材料类型与工况条件的磁芯损耗预测模型。该模型极大提升了磁性元件设计的精确性与效率,为电力电子技术的进一步发展奠定坚实基础。

请利用附件一中的实验数据,通过数据分析与建模技术,构建磁芯损耗预测模型,分析模型的预测精度、泛化能力,以及对业界的各种指导意义;同时对附件三中样本的磁芯损耗进行预测,把预测结果填入附件四(Excel表格)中第3列,要求:
\begin{enumerate}
    \item 按样本序号填入相应的磁芯损耗预测结果,只保留小数点后1位;
    \item 结果填入附件四后,保留原文件名(与问题一的结果一起)以附件材料上传;
    \item 特别把附件三中样本序号为:16, 76, 98, 126, 168, 230, 271, 338, 348, 379 的磁芯损耗预测结果,以表格形式呈现在论文正文中。
\end{enumerate}

\section{问题五:磁性元件的最优化条件}

在磁性元件的设计与优化领域内,磁芯损耗固然是一个不容忽视的核心评价指标,但在工程实践中,为了实现磁性元件整体性能的持续最优化,需要综合考虑多个评价指标,其中,传输磁能就是重要的评价指标之一。因此,同时考虑磁芯损耗与传输磁能这两个评价指标,对于指导磁性元件的设计方向,优化其性能表现,具有重要的理论及实践意义。

请以问题四构建的磁芯损耗预测模型为目标函数,同时考虑传输磁能这个重要指标(由于传输磁能的复杂性,我们以频率与磁通密度峰值的乘积来衡量传输磁能大小),利用附件一中的实验数据,建立优化模型,分析在什么条件下(温度、频率、波形、磁通密度峰值及磁芯材料),能达到最小的磁芯损耗以及具有最大的传输磁能(即 $f \times B_m$ 达到最大)?



\section*{参考文献}
\begin{enumerate}
    \item G. Bertotti. General properties of power losses in soft ferromagnetic materials. IEEE Transactions on Magnetics, 1988: 621-630.
    \item C. P. Steinmetz. On the law of hysteresis. AIEE Transactions, vol. 9, 1892: 3-64.
\end{enumerate}

\section*{备注 1:磁芯损耗模型的相关知识}

以下更详细介绍二类磁芯损耗模型的相关知识:

\subsection{1. 磁芯损耗分离模型}

基于磁芯损耗的物理机制,将总损耗分解为三个独立的成分(见公式(1))。

\subsubsection{(1) 磁滞损耗}

在磁化过程中,磁芯材料内部的磁畴展现出不同的响应机制。其中,一部分磁畴与外加磁场方向相近,它们通过一种类似“弹性”的灵活转动来顺应磁场变化,这种顺应在撤除外加磁场后,使得磁畴仍能维持其被磁化的方向。而另一部分磁畴则需要跨越较大的障碍,克服磁畴壁之间的摩擦力,实现一种更为“刚性”的转动,当外加磁场消失时,这些经历“刚性”转动的磁畴同样维持其磁化方向不变。因此,磁化过程中输入给磁场的能量被划分为两大类别:第一类能量,如同蓄水池中的势能,当磁化电流切断时,能量回流至电路中;而第二类能量,则转化为克服磁畴间摩擦所消耗的热能,称为磁滞损耗,其结果是磁芯材料因发热而散失能量。

特别地,单位体积内的磁滞损耗量,直接与静态磁滞回线所围成的面积成正比,这是一个重要的量化指标。在低频工作环境下,磁滞损耗相较于其他两种损耗(如涡流损耗和剩余损耗)显得尤为显著,占据主导地位,对系统的整体性能产生重要影响。磁滞损耗计算如下:

\begin{equation}
P_h = k_h f B_m^\beta
\end{equation}

其中,$f$ 是频率,$B_m$ 是磁通密度的峰值(最大值),$k_h$ 和 $\beta$ 是通过实验数据拟合出的系数。

\subsubsection{(2) 涡流损耗}

磁芯材料虽然具备较高的电阻率,但并非理想中的无穷大值。因此,当磁通交变时,会在磁芯中产生感应电压,进而产生电流(通常称为涡流)。在磁芯中内部循环,导致能量的耗散,称为涡流损耗。计算如下:

\begin{equation}
P_{cl} = k_{cl} f^2 B_m^2
\end{equation}

其中,$f$ 是频率,$B_m$ 是磁通密度的峰值;$k_{cl}$ 则与被测磁芯的截面积与电阻率有关。

\subsubsection{(3) 剩余损耗}

除了磁滞损耗和涡流损耗以外的损耗统称为剩余损耗。剩余损耗取决于材料的固有特性之中,具体包括:磁畴壁及其相互间角度的动态形成与变化;磁畴壁运动过程中的非正弦性、不均匀性以及这种运动模式的重复性;磁通密度表现出的非正弦波形及其在空间上的局部波动;材料内部粒子间复杂而微妙的相互作用力;以及磁畴壁在特定条件下发生的成核与消失等微观现象。剩余损耗的计算如下:

\begin{equation}
P_e = 8\sigma \cdot S \cdot G \cdot V_0 \cdot f^{1.5} \cdot B_m^{1.5}
\end{equation}

其中,$\sigma$ 是电导率;$S$ 是磁性元件的有效截面积;$G$ 是一个常数 0.1356;$V_0$ 是需拟合的系数。

\subsection{2. 磁芯损耗经验计算模型}

常见经验计算模型有以下三种:

\subsubsection{(1) 斯坦麦茨方程 (Steinmetz-equation)}

见公式(2)。

\subsubsection{(2) 修正的斯坦麦茨方程}

电力电子功率变换器中,磁性元件的工作励磁绝大部分为非正弦波。对于非正弦波励磁下的磁芯损耗,仅依靠 SE 方程来计算会造成较大的误差。文献[3]在斯坦麦茨方程的基础上,不增加额外的参数,提出了修正的斯坦麦茨方程(Modified Steinmetz-Equation, MSE)来计算任意波形励磁下的磁芯损耗。MSE 假设磁芯损耗与磁通密度变化率有关的物理性质,再通过对不同时刻的磁通密度变化率加权处理,得到加权平均磁密变化率,并认为磁芯损耗由加权平均磁密变化率决定。从而计算出任意波形励磁下的等效正弦波频率:

\begin{equation}
f_{\text{sin.eq}} = \frac{2}{\Delta B^2 \cdot \pi t_0} \int_0^T \left( \frac{dB(t)}{dt} \right)^2 dt
\end{equation}

而后用等效正弦波频率计算非正弦波励磁下磁芯损耗:

\begin{equation}
P = f \cdot \left( k_2 f_{\text{sin.eq}}^{\alpha_2 - 1} B_m^{\beta_2} \right)
\end{equation}

其中,$f$ 是非正弦波频率;$k_2$, $\alpha_2$, $\beta_2$ 是等效频率下斯坦麦茨方程的系数与公式(5)的系数获取方法一致。

\subsubsection{(3) IGSE (Improved Generalized Steinmetz Equation)}

针对任意波形励磁下磁芯损耗的计算,文献[4]提出了 IGSE 模型,认为磁芯损耗与磁通密度变化率 $\frac{dB(t)}{dt}$ 以及磁通密度在一个周期内的峰峰值 $\Delta B$ 均存在相关性,如式(8)、(9)所示:

\begin{equation}
P = \frac{k_i (\Delta B)^{\beta - \alpha}}{T} \int_0^T \left| \frac{dB}{dt} \right|^\alpha dt
\end{equation}

\begin{equation}
k_i = \frac{k}{(2\pi)^{\alpha-1} \int_0^{2\pi} \left| \cos \theta \right|^\alpha 2^{\beta - \alpha} d\theta}
\end{equation}

其中:$\Delta B$ 是磁通密度在一个周期内的峰峰值(即一个周期内最大值与最小值的差值);$k$, $\alpha$, $\beta$ 是等效频率下斯坦麦茨方程的系数。

\documentclass{article}
\usepackage{amsmath}
\usepackage{graphicx}

\title{数据驱动下磁性元件的磁芯损耗建模问题解答}
\author{}
\date{}

\begin{document}

\maketitle

\section{简要概括两个磁性元件磁芯损耗的计算模型的区别}

磁芯损耗的计算模型主要分为两类:

1. **损耗分离模型**:将磁芯损耗分解为三个部分:
   \begin{equation}
   P_{\text{core}} = P_h + P_{cl} + P_e
   \end{equation}
   其中,$P_h$ 为磁滞损耗,$P_{cl}$ 为涡流损耗,$P_e$ 为剩余损耗。该模型基于物理机制,细致描述了不同损耗的来源。

2. **斯坦麦茨方程**:是一种经验模型,基于实验数据拟合出损耗与频率 $f$ 和磁通密度峰值 $B_m$ 的幂函数关系:
   \begin{equation}
   P = k_1 f^{\alpha_1} B_m^{\beta_1}
   \end{equation}
   其中,$k_1$、$\alpha_1$ 和 $\beta_1$ 是经验系数,适用于正弦波励磁。

---

\section{数据集格式与含义}

数据集的格式为:
\begin{itemize}
    \item 第1列:温度(Temperature),单位:摄氏度(°C)。
    \item 第2列:频率(Frequency),单位:赫兹(Hz)。
    \item 第3列:磁芯损耗(Core Loss),单位:瓦特每立方米(W/m³)。
    \item 第4列:励磁波形类型(Excitation Waveform),类型包括正弦波、三角波、梯形波。
    \item 第5列至第1028列:磁通密度(Magnetic Flux Density, B),共1024个采样点,单位为特斯拉(T)。
\end{itemize}

---

\section{输出磁芯损耗的系统/函数入参}

系统需要以下入参来预测磁芯损耗:
\begin{itemize}
    \item 温度 $T$ (摄氏度,°C)
    \item 频率 $f$ (赫兹,Hz)
    \item 磁通密度峰值 $B_m$ (特斯拉,T)
    \item 励磁波形类型(正弦波、三角波、梯形波)
    \item 磁通密度波形数据 $B(t)$
    \item 磁芯材料类型
\end{itemize}

---

\section{专业名词的英文缩写及Python变量名}

\begin{itemize}
    \item 温度(Temperature, °C): \texttt{temp\_C}
    \item 频率(Frequency, Hz): \texttt{freq\_Hz}
    \item 磁芯损耗(Core Loss, W/m³): \texttt{core\_loss\_Wm3}
    \item 励磁波形(Excitation Waveform): \texttt{waveform\_type}
    \item 磁通密度 B(Magnetic Flux Density B, T): \texttt{flux\_density\_B}
\end{itemize}

---

\section{斯坦麦茨方程适用条件}

斯坦麦茨方程适用于以下条件:
\begin{itemize}
    \item 正弦波励磁条件下。
    \item 系数 $k_1$、$\alpha_1$、$\beta_1$ 适用于特定的磁芯材料。
    \item 在较稳定的温度和工况下,适合中低频范围内的磁芯损耗预测。
\end{itemize}

---

\section{EDA的首要任务}

在数据分析的初步阶段,EDA 会首先:
\begin{itemize}
    \item 理解数据结构、查看列名和数据类型。
    \item 检查缺失值、异常值,处理重复数据。
    \item 可视化各变量的分布,绘制磁通密度波形的图表。
    \item 计算相关性,分析不同变量之间的关系。
\end{itemize}

---

\section{机器学习方法的适用性}

机器学习适用于本问题,尤其是在以下问题中:
\begin{itemize}
    \item **励磁波形分类**:多分类问题,可以使用决策树、支持向量机等分类算法。
    \item **磁芯损耗预测**:回归问题,可以使用线性回归、随机森林、神经网络等模型预测磁芯损耗。
\end{itemize}

---

\section{交流功率法的借鉴意义}

交流功率法对问题1至5有重要借鉴意义。该方法通过测量励磁电流和感应电压,计算磁芯损耗,为模型的验证、数据采集、特征提取和误差分析提供了标准依据。这有助于理解损耗的物理机制,支持损耗模型的构建与修正。

---

\section{磁性材料的不同带来的影响}

磁芯材料的不同会显著影响损耗预测。不同材料具有不同的磁性能参数,如磁导率、饱和磁通密度等,可能导致训练集中的模型无法泛化到测试集中的新材料,预测误差增大。解决方案是增加训练集的材料多样性,或对材料特性进行建模。

---

\section{是否为时间序列问题}

此问题不属于典型的时间序列问题。虽然磁通密度 $B(t)$ 随时间变化,但每个样本的损耗是基于一个完整周期的数据,损耗之间没有跨周期的依赖关系,因此无需考虑上一时刻的损耗对下一时刻的影响。

---

\section{识别波形的数值特征判断方法}

可以通过以下特征来判断励磁波形:
\begin{itemize}
    \item **斜率分析**:正弦波斜率连续变化,三角波斜率恒定,梯形波有平坦段。
    \item **峰值与谷值计数**:正弦波和三角波每周期各有一个峰值和谷值,梯形波则可能存在平坦区域。
    \item **傅里叶变换**:分析频域特征,正弦波频谱单一,三角波和梯形波有较多谐波。
\end{itemize}

---

\section{温度修正的可能性及数据分析直觉}

可以在斯坦麦茨方程中通过以下方式增加温度修正项:
\begin{itemize}
    \item 使系数 $k_1$、$\alpha_1$、$\beta_1$ 随温度 $T$ 变化。
    \item 或增加温度修正项,如 $P = k_1 f^{\alpha_1} B_m^{\beta_1} T^\gamma$。
\end{itemize}
从数据中获取直觉的步骤:
\begin{itemize}
    \item 绘制温度与损耗的关系图。
    \item 通过拟合模型观察温度对损耗系数的影响。
\end{itemize}

---

\section{附件一中是否含有“磁芯材料”这一列}

附件一中没有单独的“磁芯材料”这一列,材料信息是通过不同数据表的命名区分的。在数据整合时,可以人为添加这一列。

---

\section{操作步骤:分析温度、波形和材料对磁芯损耗的独立及协同影响}

1. 数据合并:将不同材料的数据表合并,并标注材料信息。
2. 特征提取:从磁通密度中提取峰值、波形特征。
3. 独立影响分析:使用可视化和回归分析,评估各因素的独立影响。
4. 协同影响分析:通过方差分析(ANOVA)或交互作用图,分析温度、波形、材料之间的协同效应。
5. 寻找最优条件:通过最小磁芯损耗的样本筛选,确定在什么温度、波形和材料下损耗最小。

---

\section{深度机器学习的适用性与改动建议}

深度学习适用于本问题的回归建模,特别是在处理复杂的波形数据时具有优势。代码改动建议:
\begin{itemize}
    \item **数据预处理**:调整数据格式,标准化输入。
    \item **模型构建**:使用 CNN 或 LSTM 处理波形数据。
    \item **模型训练**:加入早停机制,调整学习率。
\end{itemize}
深度学习可以提高磁芯损耗预测的精度,但也需注意数据规模和模型泛化能力。

---

\section{多目标优化问题:最小磁芯损耗与最大传输磁能的冲突}

最小磁芯损耗和最大传输磁能两者存在冲突。传输磁能 $E = f \times B_m$ 需要高频率和高磁通密度峰值,而增加频率 $f$ 和 $B_m$ 会导致损耗增加,无法同时优化。解决这一冲突可以通过多目标优化算法(如 NSGA-II)找到帕累托最优解。优化自变量包括温度、频率、磁通密度峰值、波形类型和材料。

\end{document}


\end{document}
